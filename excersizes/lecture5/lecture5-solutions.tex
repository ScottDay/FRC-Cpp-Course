\documentclass[12pt]{article}

\usepackage[margin=1in]{geometry}
\usepackage{amsmath,amsthm,amssymb}
\usepackage{pgf-umlcd}
\usepackage{mdframed}
\usepackage[cache=false]{minted}
\usepackage{tcolorbox}

\tcbuselibrary{minted, skins, listings}

\definecolor{cppcodebg}{rgb}{0.20,0.20,0.20} %0.85

\setmintedinline[cpp]{style=scott, bgcolor=black!70, fontsize=\footnotesize}

\newtcblisting{cppcode}[1][]{
    listing engine=minted,
    colback=cppcodebg,
    colframe=black!70,
    listing only,
    size=tight,
    minted style=scott,
    minted language=cpp,
    minted options={
        breaklines=true,
        fontsize=\tiny,
        tabsize=2,
        linenos=true,
        numbersep=2mm,
        texcl=true,
        #1
    },
    left=5mm,
    enhanced,
    overlay={\begin{tcbclipinterior}\fill[black!25] (frame.south west)
            rectangle ([xshift=5mm]frame.north west);\end{tcbclipinterior}}
}

\newcommand{\N}{\mathbb{N}}
\newcommand{\R}{\mathbb{R}}
\newcommand{\Z}{\mathbb{Z}}
\newcommand{\Q}{\mathbb{Q}}

\newenvironment{theorem}[2][Theorem]{\begin{trivlist}
\item[\hskip \labelsep {\bfseries #1}\hskip \labelsep {\bfseries #2.}]}{\end{trivlist}}
\newenvironment{lemma}[2][Lemma]{\begin{trivlist}
\item[\hskip \labelsep {\bfseries #1}\hskip \labelsep {\bfseries #2.}]}{\end{trivlist}}
\newenvironment{exercise}[2][Exercise]{\begin{trivlist}
\item[\hskip \labelsep {\bfseries #1}\hskip \labelsep {\bfseries #2.}]}{\end{trivlist}}
\newenvironment{problem}[2][Problem]{\begin{trivlist}
\item[\hskip \labelsep {\bfseries #1}\hskip \labelsep {\bfseries #2.}]}{\end{trivlist}}
\newenvironment{question}[2][Question]{\begin{trivlist}
\item[\hskip \labelsep {\bfseries #1}\hskip \labelsep {\bfseries #2.}]}{\end{trivlist}}
\newenvironment{corollary}[2][Corollary]{\begin{trivlist}
\item[\hskip \labelsep {\bfseries #1}\hskip \labelsep {\bfseries #2.}]}{\end{trivlist}}

\begin{document}

\title{Weekly Excersize 5 - Object Orientation}
\author{Scott Day\\
Curtin FRC: Pre-season Training}

\maketitle

% --------------------

\begin{flushleft}
The objectives of this weeksheet are:

\begin{itemize}
    \item Implement the basic object orientation concepts.
    \item Understand how a class is dependant on its class fields.
    \item Smash all the things.
\end{itemize}
\end{flushleft}

% -------------------------------------------------------------------

\begin{problem}{1}
\text{ }\\
Write a class having two private variables and one member function which will return the area of the rectangle
\end{problem}

\begin{proof}
\begin{cppcode}[]
#include <iostream>
using namespace std;

class CRectangle
{
    public:
        int x, y;

        int area()
        {
            return x * y;
        }
};

int main ()
{
    CRectangle rect;
    cout << "Enter length of rectangle: ";
    cin >> rect.x;
    cout << "Enter breadth of rectangle: ";
    cin >> rect.y;
    cout << "Area: " << rect.area();

    return 0;
}
\end{cppcode}
\end{proof}
\pagebreak

% --------------------

\begin{problem}{2}
\text{ }\\
Write a program that inputs two integers in main and passes them to the default constructor of a class. Show the result of the addition of two numbers.
\end{problem}

\begin{proof}
\begin{cppcode}[]
#include <iostream>
using namespace std;

class Data
{
    public:
        int num1, num2;

        Data(int num1, int num2)
        {
            this.num1 = num1;
            this.num2 = num2;
            cout << "numbers initialized" << endl;
        }

        int sum()
        {
            return num1 + num2;
        }
};

int main ()
{
    int num1, num2;

    cout << "Enter first number: ";
    cin >> num1;

    cout << "Enter second number: ";
    cin >> num2;

    Data data(num1, num2);

    cout << "The addition result on: " << data.sum() << endl;

    return 0;
}
\end{cppcode}
\end{proof}
\pagebreak

% --------------------

\begin{problem}{3}
\text{ }\\
Write a C++ class called 'Student' with the data members:
\begin{itemize}
    \item name (char type)
    \item marks1, marks2 (integer type)
\end{itemize}

\text{ }\\
The program asks the user to enter name and marks. Then calcMedia() calculates the media note and disp() display name and total media mark on screen in different lines.
\end{problem}

\begin{proof}
\begin{cppcode}[]
#include <iostream>
using namespace std;

class Student
{
    public:
        char* name;
        int mark1, mark2;

        Student(char* name, int mark1,int mark2)
        {
            this.name = name;
            this.mark1 = mark1;
            this.mark2 = mark2;
        }

        int calcMedia()
        {
            return (mark1 + mark2) / 2;
        }

        void display()
        {
            cout << "Student: " << name << endl;
            cout << "Media: " << calcMedia() << endl;
        }
};

int main()
{
    char* name;
    int mark1, mark2;

    cout << "Enter name: ";
    cin >> name;
    cout << "Enter marks of two subjects: ";
    cin >> mark1;
    cin >> mark2;
    Student student1(name, mark1, mark2);
    student1.display();

    return 0;
}
\end{cppcode}
\end{proof}
\pagebreak

% --------------------

\begin{problem}{4}
\text{ }\\
Perform addition operation on complex data using class and object. The program should ask for real and imaginary part of two complex numbers (doubles), and display the real and imaginary parts of their sum.
\end{problem}

\begin{proof}
\begin{cppcode}[]
#include <iostream>
using namespace std;

class Imagin
{
	public:
    	double x, y;

    	Imagin(double x, double y)
        {
    	    this.x = x;
            this.y = y;
        }
};

int main()
{
	double num1, num2, num3, num4, real, imagin;

	cout << "First number" << endl;
	cout << "Enter the real part: ";
	cin >> num1;
	cout << "Enter the imaginary part: ";
	cin >> num2;

    cout << "Second number" << endl;
	cout << "Enter the real part: ";
	cin >> num3;
	cout << "Enter the imaginary part: ";
	cin >> num4;

	Imagin number1(num1, num2);
	Imagin number2(num3, num4);

    real = number1.x + number2.x;
    imagin = number1.y + number2.y;

    cout << "The sum of the real parts is " << real << endl;
    cout << "The sum of the imaginary parts is " << imagin << endl;

    return 0;
}
\end{cppcode}
\end{proof}

% --------------------

\end{document}
