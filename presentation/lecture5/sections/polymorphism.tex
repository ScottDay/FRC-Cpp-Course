\documentclass[../lecture5-objectorientation.tex]{subfiles}

\begin{document}

\section{Polymorphism}

% -------------------------------------------------------------------

\begin{frame}[fragile]{Polymorphism in C++}
    The word polymorphism means having many forms. Typically, polymorphism occurs when there is a hierarchy of classes and they are related by inheritance. \newline

    C++ polymorphism means that a call to a member function will cause a different function to be executed depending on the type of object that invokes the function.
\end{frame}

\begin{frame}[fragile]{Example}
    \begin{columns}[T,onlytextwidth]
        \column{.5\textwidth}
            \begin{cppcode}[lastline=34]
#include <iostream>
using namespace std;

class Shape
{
    protected:
        int width, height;

    public:
        Shape(int a = 0, int b = 0)
        {
            width = a;
            height = b;
        }

        virtual int area()
        {
            cout << "Parent class area: " << endl;
            return 0;
        }
};

class Rectangle: public Shape
{
    public:
        Rectangle( int a = 0, int b = 0):Shape(a, b)
        { }

        int area ()
        {
            cout << "Rectangle class area: " << endl;
            return width * height;
        }
};
            \end{cppcode}
        \column{.5\textwidth}
            \begin{cppcode}[lastline=34]
class Triangle: public Shape
{
    public:
        Triangle(int a = 0, int b = 0):Shape(a, b)
        { }

        int area ()
        {
            cout << "Triangle class area :" << endl;
            return (width * height / 2);
        }
};

// Main function for the program
int main( )
{
    Shape *shape;
    Rectangle rec(10,7);
    Triangle  tri(10,5);

    // store the address of Rectangle
    shape = &rec;

    // call rectangle area.
    shape->area();

    // store the address of Triangle
    shape = &tri;

    // call triangle area.
    shape->area();

    return 0;
}
            \end{cppcode}
    \end{columns}
\end{frame}

% -------------------------------------------------------------------

\begin{frame}[fragile]{Virtual Function}
    Blarg blarg blarg
\end{frame}

% -------------------------------------------------------------------

\begin{frame}[fragile]{Pure Virtual Functions}
    It's possible that you'd want to include a virtual function in a base class so that it may be redefined in a derived class to suit the objects of that class, but that there is no meaningful definition you could give for the function in the base class. \newline

    \begin{cppcode}[lastline=16]
class Shape
{
    protected:
        int width, height;

    public:
        Shape(int a = 0, int b = 0)
        {
            width = a;
            height = b;
        }

        // pure virtual function
        virtual int area() = 0;
};
    \end{cppcode} \newline

    The = 0 tells the compiler that the function has no body and above virtual function will be called \textbf{pure virtual function}.
\end{frame}

% -------------------------------------------------------------------

\end{document}
