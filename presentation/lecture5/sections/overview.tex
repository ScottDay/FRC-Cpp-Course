\documentclass[../lecture5-objectorientation.tex]{subfiles}

\begin{document}

\section{Object Orientation}

% -------------------------------------------------------------------

\begin{frame}[fragile]{Object Orientation}
    \note{Prime purpose of C++ was to add object orientation to C, which is in itself one of the most powerful programming languages.}

    Core of Object Oriented Programming (OOP) is to create objects, in code, that have certain properties and methods. \newline

    While designing C++ modules, we try to see the whole world in the form of objects. \newline

    For example, a car is an object which has certain properties such as color, number of doors, and the like. It also has ccertain methods such as accelerate, brake, and so on.
\end{frame}

% -------------------------------------------------------------------

\begin{frame}[fragile]{Overview}
    There are a few principle concepts that form the foundation of OOP:
    \begin{description}
        \item[Object] The basic unit of OOP. Both data and function that operate on data are bundled as a unit called an \textbf{Object}.
        \item[Class]
        \item[Abstraction]
        \item[Encapsulation]
        \item[Inheritance]
        \item[Polymorphism]
        \item[Overloading]
    \end{description}
\end{frame}

% -------------------------------------------------------------------

\begin{frame}[fragile]{}
t
\end{frame}

% -------------------------------------------------------------------

\end{document}
