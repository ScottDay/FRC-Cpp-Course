\documentclass[../lecture5-objectorientation.tex]{subfiles}

\begin{document}

\section{Classes and Objects}

% -------------------------------------------------------------------

\begin{frame}[fragile]{C++ Class Definitions}
    A class definition starts with the keyword \textbf{class} followed by the class name; and the class body, enclosed by a pair of curly braces. Then a semicolon or a list of declarations.

    \begin{cppcode}[lastline=7]
class Box
{
    public:
        double length;  // Length of a box
        double breadth; // Breadth of a box
        double height;  // Height of a box
}
    \end{cppcode}

    The keyword \textbf{public} determines the access attributes of the members of the class that follow it. You can also specify it as either \textbf{private} or \textbf{protected}.
\end{frame}

% -------------------------------------------------------------------

\begin{frame}[fragile]{Define C++ Objects}
    We declare objects of a class with exatly the same sort of declaration that we declare variables of basi types.

    \begin{cppcode}[lastline=2]
Box box1;
Box box2;
    \end{cppcode}
\end{frame}

% -------------------------------------------------------------------

\begin{frame}[fragile]{Accessing Data Members}
    Public data members of objects of a class can be accessed using the direct member access operator (.).

    \begin{columns}[T,onlytextwidth]
        \column{.5\textwidth}
            \begin{cppcode}[lastline=25]
#include <iostream>

using namespace std;

class Box {
   public:
      double length;  // Length of a box
      double breadth; // Breadth of a box
      double height;  // Height of a box
};

int main() {
   Box box1;            // Declare Box1 of type Box
   Box box2;            // Declare Box2 of type Box
   double volume = 0.0; // Store the volume of a box here

   // box 1 specification
   box1.height  = 5.0;
   box1.length  = 6.0;
   box1.breadth = 7.0;

   // box 2 specification
   box2.height  = 10.0;
   box2.length  = 12.0;
   box2.breadth = 13.0;
            \end{cppcode}
        \column{.5\textwidth}
            \begin{cppcode}[lastline=10]
   // volume of box 1
   volume = box1.height * box1.length * box1.breadth;
   cout << "Volume of Box1 : " << volume << endl;

   // volume of box 2
   volume = box2.height * box2.length * box2.breadth;
   cout << "Volume of Box2 : " << volume << endl;

   return 0;
}
            \end{cppcode}
    \end{columns}
\end{frame}

% -------------------------------------------------------------------

\begin{frame}[fragile]{Classes \& Objects in Detail}
    So far, we have covered the very basics of C++ Classes and Objects. Here are some further concepts we will discuss at some point.
\end{frame}

\begin{frame}[fragile]{Classes \& Objects in Detail}
    \begin{table}[h]
        \center
        \begin{tabularx}{\textwidth}{c X}
            \textbf{Class Member Functions} & Functions with it's definition/prototype within the class definition like any other variable. \\[0.5cm]
            \textbf{Class Access Modifiers} & Class members defined as public, private or protected. \\[0.5cm]
            \textbf{Constructor} & Special function in a class that's called when a new object of the class is created. \\[0.5cm]
            \textbf{Destructor} & Special function which is called when a created object is deleted. \\[0.5cm]
            \textbf{Copy Constructor} & Creates an object by initializing it with an object of the same class, which has been created previously. \\
        \end{tabularx}
    \end{table}
\end{frame}

\begin{frame}[fragile]{Classes \& Objects in Detail}
    \begin{table}[h]
        \center
        \begin{tabularx}{\textwidth}{c X}
            \textbf{Friend Functions} & Function that is permitted full access to private and protected members of a class. \\[0.5cm]
            \textbf{Inline Functions} & The compiler tries to expand the code in the body of the function in place of a call to the function. \\[0.5cm]
            \textbf{this pointer in C++} & Every object has a special pointer \textbf{this} which points to the object itself. \\[0.5cm]
            \textbf{Pointer to C++ Classes} & A pointer to a class done in the same way a pointer to a structure is. \\[0.5cm]
            \textbf{Static Members of a Class} & Both data and function members of a class can be declared as static. \\
        \end{tabularx}
    \end{table}
\end{frame}

% -------------------------------------------------------------------

\end{document}
