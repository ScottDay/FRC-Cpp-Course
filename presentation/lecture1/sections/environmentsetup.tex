\documentclass[../lecture1-introduction.tex]{subfiles}

\begin{document}

\section{Environment Setup}

% -------------------------------------------------------------------

\begin{frame}[fragile]{C++ Editors}
    The easiest way to compile console programs depeds on the particular tool you
    are using. \newline

    The easiest way for beginners to compile C++ programs is by using an Integrated
    Development Environment (IDE). \newline

    An IDE generally integrates several development tools, including a text editor
    and tools to compile programs directly from it.
\end{frame}

% -------------------------------------------------------------------

\begin{frame}[fragile]{Samples}
    Some IDE's:
    \begin{itemize}
        \item \href{https://www.visualstudio.com/}{Visual Studio},
        \item \href{https://www.jetbrains.com/clion/}{CLion}
    \end{itemize}

    Or you can use a text editor:
    \begin{itemize}
        \item \href{https://atom.io/}{Atom} (pretty baller),
        \item \href{https://www.sublimetext.com/}{Sublime Text},
        \item \href{https://notepad-plus-plus.org/}{Notepad++}
    \end{itemize}

    Sample compilers:
    \begin{itemize}
        \item \href{https://gcc.gnu.org/}{GCC} (use MinGW for windows),
        \item \href{http://clang.llvm.org/}{Clang}
    \end{itemize}
\end{frame}

% -------------------------------------------------------------------

\begin{frame}[fragile]{Setup and Installation}
    Refer to the relevant documentation of whatever tool/compiler you choose to use.
\end{frame}

% -------------------------------------------------------------------

\begin{frame}[fragile]{Compiling}
    The typical filename extensions are:
    \begin{itemize}
        \item ".cpp" for a C++ source file.
        \item ".hpp" for a C++ header file.
    \end{itemize}
    Gcc can compile C++ as well as C:
    \begin{commandshell}
gcc -c file1.cpp
gcc -c file2.cpp
gcc file1.o file2.o -o prog -lstdc++
    \end{commandshell}
    \begin{itemize}
        \item The ".cpp" extension tells gcc that it's dealing with C++ code.
        \item "-o name" gives the output filename. Without it the executable will be
        called "a.out", which is silly.
        \item "-lstdc++" tells gcc to link against the C++ library.
    \end{itemize}
    Alternatively, you can use "g++":
    \begin{commandshell}
g++ file1.o file2.o -o prog
    \end{commandshell}
    You can use Makefiles to simplify the process.
\end{frame}

% -------------------------------------------------------------------

\begin{frame}[fragile]{Makefiles}
    \begin{cppcode}
target: dependencies
[tab] system command
    \end{cppcode}
    This syntax applied to our example would look like:
    \begin{cppcode}
all:
    g++ file1.cpp file2.cpp -o hello
    \end{cppcode}
    To run this makefile, type:
    \begin{commandshell}
make
    \end{commandshell}
\end{frame}

% -------------------------------------------------------------------

\begin{frame}[fragile]{Lets Just Remember the Cats}
    \begin{center}
        \makebox[\textwidth]{\includegraphics[width=\paperwidth,height=\textheight,keepaspectratio]{graphics/breading-cats-2.jpg}}
    \end{center}
\end{frame}

% -------------------------------------------------------------------

\end{document}
