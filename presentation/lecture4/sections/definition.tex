\documentclass[../lecture4-functions.tex]{subfiles}

\begin{document}

\section{Function Definition}

% -------------------------------------------------------------------

\begin{frame}[fragile]{}
    In C++, a function is a group of statements that is given a name, and which can be called from some point of the program. The most common syntax to define a function is: \newline

    % fix this
    return-value-type function-name(parameter1, parameter2, \dots) \newline
    \{ \newline
        \quad declaration of local variables; \newline
        \quad statements; \newline

        \quad return return-value; \newline
    \} \newline

    Where:
    \begin{description} % Replace with table, bold left column, no vertical line
        \item[return-value-type] Is the type of the value returned by the function.
        \item[function-name]     Is the identifier by which the function can be called.
        \item[parameters] Each parameter consists of a type followed by an identifier, with each parameter beign separated from the next by a comma.
        \item[statements] Is the function's body. It is a block of statements surrounded by braces \{ \} that specify what the function actually does.
    \end{description}
\end{frame}

% -------------------------------------------------------------------

\begin{frame}[fragile]{}
t
\end{frame}

% -------------------------------------------------------------------

\begin{frame}[fragile]{}
t
\end{frame}

% -------------------------------------------------------------------

\end{document}
