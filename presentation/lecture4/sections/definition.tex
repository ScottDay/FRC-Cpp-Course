\documentclass[../lecture4-functions.tex]{subfiles}

\begin{document}

\section{Function Definition}

% -------------------------------------------------------------------

\begin{frame}[fragile]{}
    In C++, a function is a group of statements that is given a name, and which can be called from some point of the program. The most common syntax to define a function is: \newline

\begin{cppcode}
returnValueType functionName(type parameter1, type parameter2, ...)
{
    statements;
}
\end{cppcode}

    \vspace{0.5cm}
    Where:
    \begin{tabular}{rl}
        \textbf{return-value-type} & Is the type of the value returned by the function. \\
        \textbf{function-name}     & Is the identifier by which the function can be called. \\
        \textbf{parameters}        & Each parameter consists of a type followed by an \\
                                   & identifier,  with each parameter being separated from the \\
                                   & next by a comma. \\
        \textbf{statements}        & Is the function's body. It is a block of statements \\
                                   & surrounded by braces \{ \} that specify what the \\
                                   & function actually does. \\
    \end{tabular}
\end{frame}

% -------------------------------------------------------------------

\begin{frame}[fragile]{Cute Dog}
    \begin{center}
        \makebox[\textwidth]{\includegraphics[width=0.80\paperwidth,height=0.80\textheight]{graphics/cutedog1.jpg}}
    \end{center}
\end{frame}


% -------------------------------------------------------------------

\end{document}
