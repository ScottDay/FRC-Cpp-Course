\documentclass[../lecture4-functions.tex]{subfiles}

\begin{document}

\section{Example of Function Definition, Declaration and Call}

% -------------------------------------------------------------------

\begin{frame}[fragile]{Example 1 - Addition}
    \begin{itemize}
        \item A very basic example of using a function
        \item Note: Both examples do the exact same thing.
    \end{itemize}

    \begin{columns}[T,onlytextwidth]
        \column{.5\textwidth}
            \begin{cppcode}[]
// Function Example
#include <iostream>
using namespace std;

int addition(int a, int b)
{
    int result;

    result = a + b;

    return result;
}

int main()
{
    int temp;

    temp = addition(5, 3);
    cout << "The result is: " << temp << endl;

    return 0;
}
            \end{cppcode}
        \column{.5\textwidth}
                \begin{cppcode}[]
// Function Example
#include <iostream>
using namespace std;

int addition(int a, int b)
{
    return a + b;
}

int main()
{
    cout << "The result is: " << addition(5, 3) << endl;

    return 0;
}
                \end{cppcode}
    \end{columns}
\end{frame}

% -------------------------------------------------------------------

\begin{frame}[fragile]{Example 2 - Factorial}
    Let us first look at an example of a program writen entirely with the function main() and then we will modify it to use an additional \textbf{function call}. \newline

    We will illustrate this with a program to calculate the factoriall (n!) of an integer number (n) using a for loop to compute:

    \verb|n! = 1 x 2 x 3...(n-2) x (n-1) x n|
\end{frame}

% -------------------------------------------------------------------

\begin{frame}[fragile]{Example 2 - Factorial}
    \begin{cppcode}[]
// Program to calculate factorial of a number
#include <iostream>
using namespace std;

int main()
{
    int ii, number = 0, factorial = 1;

    // User input must be an integer number between 1 and 10
    while(number < 1 || number > 10)
    {
        cout << "Enter integer number (1-10) = ";
        cin >> number;
    }

    // Calculate the factorial with a FOR loop
    for(ii = 1; ii <= number; ii++)
    {
        factorial = factorial * ii;
    }

    // Output result
    cout << "Factorial = " << factorial << endl;
    return 0;
}
    \end{cppcode}

    Even though the program is very short, the code to calculate the factorial is best placed inside a function since it is likely to be executed many times in the same program.
\end{frame}

% -------------------------------------------------------------------

\begin{frame}[fragile]{Example 2 - Factorial}
    \begin{columns}[T,onlytextwidth]
        \column{.5\textwidth}
            \begin{cppcode}[]
// Program to calculate factorial of a number with function call
#include <iostream>
using namespace std;

// Function declaration (prototype)
int factorial(int number);

int main()
{
    int number = 0, result;

    // User input must be an integer number between 1 and 10
    while(number < 1 || number > 10)
    {
       cout << "Integer number = ";
       cin >> number;
    }

    // Function call and assignment of return value to result
    result = factorial(number);

    // Output result
    cout << "Factorial = " << result << endl;
    return 0;
}
            \end{cppcode}
        \column{.5\textwidth}
            \begin{cppcode}[firstnumber = 27]
// An integer, number, is passed from caller function.
int factorial(int number)
{
    int ii, factorial = 1;

    // Calculate the factorial with a FOR loop
    for(ii = 1; ii <= number; ii++)
    {
        factorial = factorial * ii;
    }

    return factorial; // This value is returned to caller
}
            \end{cppcode}
    \end{columns}
\end{frame}

% -------------------------------------------------------------------

\begin{frame}[fragile]{Example 2 - Factorial Discussion}
    Three modifications have been made to incorporate a function:
    \begin{itemize}
        \item The \textbf{declaration} of the function above \mintinline{cpp}{int main()}. \\
                The declaration (A.K.A the prototype) tells the compiler about the function and the type of data it requires and will return on completion.
        \item The \textbf{function call} in the main body of the program determines when to branch to the function and how to return the value of the data computed back to the main program.
        \item The \textbf{definition} of the function \mintinline{cpp}{int factorial(int number)} below the main program.\\
                The definition consists of a \textbf{header} which specifies how the function will interface with the main program and a \textbf{body} which lists the statements to be executed when the function is called.
    \end{itemize}
\end{frame}

% -------------------------------------------------------------------

\begin{frame}[fragile]{Remember the Cuteness - A nose so sharp it could fly}
    \begin{center}
        \makebox[\textwidth]{\includegraphics[width=0.80\paperwidth,height=0.80\textheight]{graphics/cutedog2.jpg}}
    \end{center}
\end{frame}

% -------------------------------------------------------------------

\end{document}
