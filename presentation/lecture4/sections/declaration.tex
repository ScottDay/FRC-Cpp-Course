\documentclass[../lecture4-functions.tex]{subfiles}

\begin{document}

\section{Function Declaration}

% -------------------------------------------------------------------

\begin{frame}[fragile]{Function Declaration}
    Every function has to be \textbf{declared} before it us used. The declaration tells the compiler the name, \verb|return value type| and \verb|parameter types| of the funcction. \newline

    In this example the declaration:

    \begin{cppcode}[lastline = 1, fontsize=\footnotesize]
int fatorial(int number);
    \end{cppcode}

    tells the compiler that the program passes the value of an integer to the function and that the \verb|return value| must be assigned to an integer variable. The declaration of a function is ccalled its \textbf{prototype}, which means the "first" time the function is identified to your program. \newline

    The function prototype and the function definition must agree exactly about the \verb|return value type|, \verb|function name| and the \verb|parameter types|. The function prototype is usally a copy of the function header followed by a semicolon to make it a declaration and placed before the main program in the program file.
\end{frame}

% -------------------------------------------------------------------

\end{document}
