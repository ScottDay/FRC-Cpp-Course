\documentclass[../lecture3-flowofcontrol.tex]{subfiles}

\begin{document}

\section{Loops}

% -------------------------------------------------------------------

\begin{frame}[fragile]{Loops}
Conditionals execute certain statements if certa\texttt{while} in conditions are met; loops execute certain statements while certain conditions are met. C++ has three kinds of loops: \texttt{while}, \texttt{do-while}, and \texttt{for}.
\end{frame}

% -------------------------------------------------------------------

\begin{frame}[fragile]{While}
    The \texttt{while} loop has a form similar to the if conditional:

\begin{cppcode}[]
while(condition)
{
    statement1
    statement2
    ...
}
\end{cppcode}

    As long as the condition holds, the block of statements will be repeatedly executed. \newline

    If there is only one statement, the curly braces may be omitted.
\end{frame}

\begin{frame}[fragile]{While Example}
\begin{cppcode}[]
#include <iostream>
using namespace std;

int main()
{
    int x = 0;

    while(x < 10)
        x = x + 1;

    cout << "x is " << x << endl;

    return 0;
}
\end{cppcode}

    This program will print \verb|x is 10|.
\end{frame}

\begin{frame}[fragile]{Do-While}
    The \texttt{do-while} loop is a variation that guarantees the block of statements will be executed at least once:

\begin{cppcode}[]
do
{
    statement1
    statement2
    ...
}
while(condition);
\end{cppcode}

    The block of statements is eecuted and then, if the condition holds, the program returns to the top of the block. \newline

    Curly braces are always required. \newline

    Also not note the semicolon after the \texttt{while} condition.
\end{frame}

% -------------------------------------------------------------------

\begin{frame}[fragile]{For}
    The \texttt{for} loop works like the while loop but with some change in syntax:

\begin{cppcode}[]
for(initialization; condition; incrementation)
{
    statement1
    statement2
    ...
}
\end{cppcode}

    The for loop is designed to allow a counter variable that is initialized at the beginning of the loop and incremented (or decremented) on each iteration of the loop. \newline

    Curly braces may be omitted if there is only one statement.
\end{frame}

\begin{frame}[fragile]{For Example}
\begin{cppcode}[]
#include <iostream>
using namespace std;

int main()
{
    for(int x = 0; x < 10; x = x + 1) // or x++
        cout << x << endl;

    return 0;
}
\end{cppcode}

    This program will print out the values \texttt{0} through \texttt{9}. \newline

    If the counter variable is already defined, there is no need to define a new one in the initialization porrtion of the for loop,
\end{frame}

% -------------------------------------------------------------------

\end{document}
