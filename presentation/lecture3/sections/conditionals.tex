\documentclass[../lecture3-flowofcontrol.tex]{subfiles}

\begin{document}

\section{Conditionals}

% -------------------------------------------------------------------

\begin{frame}[fragile]{Conditionals}
t
\end{frame}

% -------------------------------------------------------------------

\begin{frame}[fragile]{Operators}
t
\end{frame}

\begin{frame}[fragile]{Relational Operators}
    \begin{table}
        \center
        \begin{tabular}{c|c|c}
            \toprule
            \textbf{Operator} & \textbf{Shorthand} & \textbf{Meaning} \\
            \midrule
            \verb|>|  &       & Greater than \\
            \verb|>=| & $\ge$ & Greater than or equal to \\
            \verb|<|  &       & Less than \\
            \verb|<=| & $\le$ & Less than or equal to \\
            \verb|==| &       & Equal to \\
            \verb|!=| & $\ne$ & Not equal to \\
            \bottomrule
        \end{tabular}
    \end{table}
\end{frame}

\begin{frame}[fragile]{Logical Operators}
    \begin{table}
        \center
        \begin{tabular}{c|c}
            \toprule
            \textbf{Operator} & \textbf{Meaning} \\
            \midrule
            \verb|&&| & and \\
            \verb+||+ & or \\
            \verb|!|  & not \\
            \bottomrule
        \end{tabular}
    \end{table}
\end{frame}

\begin{frame}[fragile]{Truth Tables}
    Operators return \verb|true| or \verb|false|, according to the rules of logic:

    \begin{columns}[T,onlytextwidth]
        \column{.2\textwidth}
            \begin{table}
                \begin{tabular}{c|c|c}
                    \toprule
                    \textbf{a} & \textbf{b} & \bfseries{a \verb|&&| b} \\
                    \midrule
                    true  & true  & true  \\
                    true  & false & false \\
                    false & true  & false \\
                    false & false & false \\
                    \bottomrule
                \end{tabular}
            \end{table}
        \column{.2\textwidth}
            \begin{table}
                \begin{tabular}{c|c|c}
                    \toprule
                    \textbf{a} & \textbf{b} & \bfseries{a \verb+||+ b} \\
                    \midrule
                    true  & true  & true  \\
                    true  & false & true  \\
                    false & true  & true  \\
                    false & false & false \\
                    \bottomrule
                \end{tabular}
            \end{table}
        \column{.2\textwidth}
            \begin{table}
                \begin{tabular}{c|c}
                    \toprule
                    \textbf{a} & \bfseries{\verb|!|a} \\
                    \midrule
                    true  & false \\
                    false & true  \\
                    \bottomrule
                \end{tabular}
            \end{table}
    \end{columns}

    \begin{center}
        Examples using logical operators (assume \verb|x = 6| and \verb|y = 2|):
    \end{center}

    \begin{table}
        \center
        \begin{tabular}{cr}
            \pause \verb|!(x > 2)|           & \pause false \\
            \pause \verb|(x > y) && (y > 0)| & \pause true  \\
            \pause \verb|(x < y) && (y > 0)| & \pause false \\
            \pause \verb+(x < y) || (y > 0)+ & \pause true  \\
        \end{tabular}
    \end{table}
\end{frame}

\begin{frame}[fragile]{C++ Boolean}
    Boolean variables can be used directly in these expressions, since they hold \verb|true| and \verb|false| values. \newline

    Funny enough, any kind of value can be used in a Boolean expression due to a quirk C++ has:
    \begin{center}
        \verb|false| is represented by a value of 0 and anything that is not 0 is \verb|true|. \newline
    \end{center}

    So, “Hello, world!” is \verb|true|, 2 is \verb|true|, and any int variable holding a non-zero value is \verb|true|. This means \verb|!x| returns \verb|false| and \verb|x && y| returns \verb|true|!
\end{frame}

% -------------------------------------------------------------------

\begin{frame}[fragile]{If, If-Else and Else-If}
t
\end{frame}

% -------------------------------------------------------------------

\begin{frame}[fragile]{Switch-Case}
t
\end{frame}

% -------------------------------------------------------------------

\end{document}
