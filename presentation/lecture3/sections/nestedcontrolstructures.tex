\documentclass[../lecture3-flowofcontrol.tex]{subfiles}

\begin{document}

\section{Nested Control Structures}

% -------------------------------------------------------------------

\begin{frame}[fragile]{Nested Control Structures}
    It is possible to place ifs inside of ifs and loops inside of loops by simply placing these structures inside the statement blocks. \newline

    This allows for more complicated program behaviour.
\end{frame}

% -------------------------------------------------------------------

\begin{frame}[fragile]{Nested If Conditionals}
\begin{cppcode}[]
#include <iostream>
using namespace std;

int main()
{
    int x = 6;
    int y = 0;

    if(x > y)
    {
        cout << "x is greater than y" << endl;

        if(x == 6)
            cout << "x is equal to 6" << endl;
        else
            cout << "x is not equal to 6" << endl;
    }
    else
        cout << "x is not greater than y" << endl;

    return 0;
}
\end{cppcode}

    This program will print \verb|x is greater than y| on one line and then \verb|x is equal to 6| on the next line.
\end{frame}

% -------------------------------------------------------------------

\begin{frame}[fragile]{Nested Loops}
\begin{cppcode}[]
#include <iostream>
using namespace std;

int main()
{
    for(int x = 0; x < 4; x = x + 1)
    {
        for(int y - 0; y < 4; y = y + 1)
            cout << y;
        cout << endl;
    }

    return 0;
}
\end{cppcode}

    This program will print four lines of \texttt{0123}.
\end{frame}

% -------------------------------------------------------------------

\end{document}
